% Documentklasse
\documentclass[a4paper,oneside,11pt]{report}

% Packages laden
\usepackage[a4paper,top=2cm,bottom=2cm,left=3cm,right=2cm]{geometry}		% paginagrootte
%\usepackage{a4wide}
\usepackage{parskip}									% andere regels voor nieuwe paragraaf: witregel + niet inspringen

\usepackage[english]{babel}						%	spelling en woordafbreking (Engles)
\usepackage[latin1]{inputenc}					% invoer van speciale tekens (bvb. Umlaut)
\usepackage[T1]{fontenc}							% weergave van speciale tekens (bvb. Umlaut)
\usepackage{lmodern}									% betere weergave van speciale tekens (bvb. Umlaut)
\usepackage{dsfont}	
\usepackage{amsfonts,amsthm, tabularx}					% wiskundige symbolen and table of equations
%\usepackage[fleqn]{amsmath}

\usepackage{graphicx,subfigure}				% figuren
\usepackage{float}										% plaatsen van figuren en tabellen
\usepackage[format=plain,
						indent=1cm]{caption}			% personaliseren van onderschriften

\usepackage{eurosym}									% sign of euro

% Instellingen voor document
\graphicspath{{Figuren/}}             % bestandslocatie van in te voegen figuren
\renewcommand{\arraystretch}{1.1}			% tabelrijen iets hoger maken

\usepackage[squaren,Gray]{SIunits}
\usepackage{amsmath,amsfonts,amsthm,mathrsfs,MnSymbol}	% wiskundige symbolen
\renewcommand*\thesection{\arabic{section}}
\DeclareMathOperator*{\argmin}{\arg\!\min}
\usepackage{pifont}							

\setcounter{secnumdepth}{3}		% Enable subsubsection numbering
\setcounter{tocdepth}{3}		% Include subsubsection in table of content

\usepackage{color}				% Load the color package: \color{declared-color}{text}. If also background:
								% \colorbox{declared-color1}{\color{declared-color2}text}
\usepackage{hyperref}


%%%%%%%%%%
%% Body %%
%%%%%%%%%%
\begin{document}

\chapter*{Borefield model}

\section{Install}
To run the model, you will need Dymola 2014 FD01 (64bit) or a later version. Make sure that you open the 64-bit by default. Also use version 3.2 or higher of the Modelica Standard Library.


To download the model, pull IDEAS from the repository on \url{https://github.com/damienpicard/IDEAS} and switch to the branch \textit{develop\_dap}. If you do not work with a version control system, go on the web page, change the branch from \textit{master} to \textit{develop\_dap} and download the zip file. Make sure your folder name is {\tt IDEAS} (and not {\tt IDEAS\_develop\_dap}). 

The model also uses the Modelica Building Library (\url{http://simulationresearch.lbl.gov/modelica/Download}). Make sure both {\tt IDEAS} as {\tt Buildings} are loaded in Dymola before running the models.

\section{Structure of the code}
The code is divided in several packages:
\begin{enumerate}
\item Data: all the parameter values of the borefield are stored here. This package also includes parameters for the calculation of the step response (short and long-term) and for the aggregation technic. The records {\tt SoilData, FillingData, GeometricData, StepResponse, Advanced} contain the general parameter values. The record {\tt ShortTermResponse} reads a mat-file which should contain the fluid temperature response for the general parameters given by the other records. Finally, the record {\tt BorefieldData} groups the six records into one. 
\item BaseClasses: most of the calculations for the borefield are done by its base classes. It includes the model for the short-term response, model for the long-term response, functions for the aggregation technic and some scripts to automate the calculation of the short-term response of given parameters and the saving of the results in a .mat file (see section \ref{sec:ini_mod}.
\item Examples: some examples to show how you can run the model
\item Extra: model for optimization (\textit{MultipleBoreholes\_signals} caluclate the in and outlet temperature and give as as an Real output signal, \textit{GroundCoupledHeatPump} combines \textit{MultipleBoreholes\_signals} with some code to calculate COP and QCond, given QEva). These models are used in \textit{Borefield/Example/borefieldWithHP}.
\end{enumerate}

\section{Run the example models}
As explained in the previous section, the record {\tt ShortTermResponse} read the short-term step-response of the borefield. The response should first be build so that you can run the examples. (There are two examples: \textit{example} and \textit{example\_accurate}). This is done using a script. You might need to change some path so that the response is saved at the right place. Do the following:

\begin{enumerate}
\item Run the script {\tt borefield/BaseClasses/Script/ShortTimeResponseHX } in order to create the step-response:
	\begin{enumerate}
	\item right click on the function name (ShortTimeResponseHX)
	\item to build the response of \textit{example}, just click on \textit{ok} without changing the inputs.
	\item Check from simulation tab that you get 3 times \textit{True}. If it is not the case, go to Debug (see below)
	\end{enumerate}
\item Debug: change \textit{savePath} to give the full path on your computer in \textit{ShortTermResponseHX} code and try again. Mind that you should use double slash.
\item Check now if {\tt borefield/Data/ShortTermResponse/example } gives no errors when you do the model check.
\item if you get an error, change the path in {\tt borefield/Data/ShortTermResponse/example } to give your full computer path.
\item it should work now
\end{enumerate}


You can now build the short term response of the example \textit{example\_accurate}. Do the following:
\begin{enumerate}
\item Run the script {\tt borefield/BaseClasses/Script/ShortTimeResponseHX } in order to create the step-response:
	\begin{enumerate}
	\item right click on the function name (ShortTimeResponseHX)
	\item fill inputs:
		\begin{enumerate}
		\item name: "example\_accurate" .
		\item Tree Data: change only the record ShortTermReponse (click on \textit{arrow > select record > example\_accurate}).
		\item Check from simulation tab that you get 3 times \textit{True}. 
		\end{enumerate}
	\end{enumerate}
\item Check now if {\tt borefield/Data/ShortTermResponse/example\_accurate } gives no errors when you do the model check.
\item if you get an error, change the path in {\tt borefield/Data/ShortTermResponse/example } to give your full computer path.
\item it should work now.
\end{enumerate}

You have now all the mat-files you need to run all examples. The last thing you need to do is to build the c-code for the error-function which is used by the some models. To do so, do the following:
\begin{enumerate}
\item if it does not yet exist, create a folder \textit{SimulationResults}. You should always be in this directory when you do a simlation.
\item copy the file \textit{erf.c} (located in {\tt IDEAS/IDEAS/Resources/CCode}) to this folder.
\item execute the function {\tt Borefield/GroundHX/BaseClasses/erf} with input \textit{u=1}. This will build the c-code to modelica-code.
\end{enumerate}

You should now be able to run all the examples of the library. Try for example the model {\tt Borefield/Example/borefieldWithHP}.

\section{Initialization of a new borefield model} \label{sec:ini_mod}
The borefield model is based on a temperature response model. Prior any simulation, the model will build a step-response and create the aggregation cells. This is only semi-automatic. Every time the user want to use a different borefield (different parameter value), he should build a new records structure as following:

\begin{enumerate}
\item Go on {\tt borefield/Data } and create new record for {\tt SoilData, FillingData, FillingData, GeometricData, StepResponse, Advanced}. Notice that most of the parameters have default values.
\item Run the script {\tt borefield/BaseClasses/Script/ShortTimeResponseHX } in order to create a new record \textit{ShortTermResponse}:
	\begin{enumerate}
	\item right click on the function name
	\item fill inputs:
		\begin{enumerate}
		\item name: give the name of your new record
		\item Tree Data: select the \textit{soi, fill, geo, adv} and \textit{steRes} that you have created (click on \textit{arrow > select record > recordName})
		\item Check from simulation tab that you get 3 times \textit{True}. 
		\item If no error, go \textit{Data/ShortTermResponse} and duplicate \textit{example} (right click > new > duplicate) and give the model a new name
		\item change the string \textit{exampleData} to the new name and click on check the model
		\item if it gives an error, give the full path of your computer
		\end{enumerate}
	\end{enumerate}
\item Finally, make a new record \textit{Data/BorefieldData} calling all the new subrecord you have made
\end{enumerate}


\section{Simulation}
Simulating the model is now very easy. Go for example on {\tt Borefield/Examples/borefieldWithHP}. Change the parameter {\tt lenSim} to the desired simulation time (necessary to know for the aggregation technique). Change also the type of the borefieldData to the newly creating record (in the model {\tt borefieldWithHP} change 

\textit{groundCoupledHeatPump(lenSim=lenSim,redeclare Borefield.Data.BorefieldData.example bfData)} to 

\textit{ groundCoupledHeatPump(lenSim=lenSim, redeclare Borefield.Data.BorefieldData.\textbf{newName} bfData)}.
\end{document}